\clearpage
\phantomsection
\chapter{Tre giorni con le ruote grasse}

\begin{flushright}
    \emph{Il viaggio è tale quando coinvolge sia la sfera fisica che quella mentale.}

    \emph{Anonimo}
\end{flushright}

\emph{Sabato $7$ agosto, ore $6$:$00$, Sluderno}  
\vspace{2mm}

Suona la sveglia e come di consueto io faccio la mia ginnastica mattutina mentre Tommaso, come al solito, guarda attonito e incredulo della mia energià già a quell'ora.
Facciamo una veloce colazione e prepariamo le ultime cose sulle notre bici.
Saranno tre giorni di nuove emozioni, non abbiamo mai fatto un viaggetto multigiorni con la mtb.

Verso le $7$:$40$ siamo in sella e cominciamo a pedalare tra le monragne della Val Venosta ai piedi dello Stelvio, al quale facciamo l'occhiolino, ci rivederemo tra $2$ giorni.
Giusto qualche km di pianura e inizia subito la salita, prima dolce e successivamente si fa più pendente, dopo una diecina di km iniziano i primi tratti sterrati che ci accompagneranno per la maggior parte dei giorni successivi.
Saliamo verso il \emph{Passo di Slingia} costeggiando una cascata che riempe l'aria di una fitta nebbiolina e ci accompagna sugli ultimi 3 km di salita che sembrano non terminare mai.
Il fondo diventa dissestato e la pendenza diventa importante e dopo un pezzo al $20/25\%$ ci vediamo costretti a smontare dalla bici e spingerla.

Arrivati al passo ci troviamo su un sentiero che sembra uscito direttamente dalle paludi del Signore degli Anelli.
Una fitta nebbia ci impedisce di vedere più in là di una cinquantina di metri e vista l'altitudine di $2309$m la temperatura è scesa fino a $10$ gradi.
Le pioggie della settimana appena passata hanno reso molti sentieri un acquitrino unico e si procede a fatica tra fango e guadi.
Quand'ecco che nel mentre stiamo passando uno dei laghetti Tommaso decide di attarversarlo senza curarsi di bagnarsi ed ad un certo punto pensa di essere finito nei "fanghi mobili".
Per un attimo sprofonda e chiama aiuto.
Dopo un attimo si rende conto che era già sul fondo e continua con l'attraversamento. Si delina così già una costante dei tre giorni: Tommaso con le scarpe bagnate e infangate mentre io cerco di evitare di immergere i piedi nei guadi.
Solo il terzo giorno mi darò per vinto, una volta sola.

Dopo una piccola discesa la nebbia di dirada e, entrati in Svizzera\footnote{Il confine tra Italia e Svizzera è posto poco prima del \emph{Passo di Slingia}.}, possiamo iniziare ad ammirare la magnificenza delle montagne che ci circonda.
Maestose e torreggianti ci incanalano nella valle d'Uina e nelle omonime gole dove è sttao scavato dai contrabbandieri un sentiero incastonto su una parete rocciosa alta $800$m.
All'inizio della gola il torrente scorre in prossimità del sentiero, ma dopo qualche decina di metri si getta in varie cascate che rendono il sentiero non adatto a persone che soffrono di vertigini.
Per un paio di km portiamo la bici a mano, il sentiero è stretto, sassaso, ripido e non lascia spazio a errori di manovra.

Al fine del sentiero roccioso si apre la valle d'Uina che porta a \emph{Sur En}, paesello dove passa il fiume \emph{En}\footnote{Il fiume è chiamato anche \emph{Inn} e prosegue per l'Austria, gettandosi infine nel Danubio} il quale nasce tra le pendici del ghiacciaio del Bernina, che raggiungeremo il giorno successivo.
Il sentiero continua in un bosco di abeti e via via che scendiamo diventa più largo e dolce.
I guadi coninuano e tra un ponte e l'altro arriviamo in fondo alla valle della bassa Valle Engandina.

\emph{Sabato $7$ agosto, ore $10$:$30$, Sur En} 
\vspace{2mm}

Giunti nel cuore del cantone dei Grigioni, dopo $3h30'$ e circa $30$km pedalati, ci fermiamo per una prima pausa a base di cocacola e paninetti al finicchio con formaggio e salame.


