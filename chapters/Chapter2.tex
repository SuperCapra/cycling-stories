\clearpage
\phantomsection
\chapter{Tre giorni con le ruote grasse}

\begin{flushright}
    \emph{Il viaggio è tale quando coinvolge sia la sfera fisica che quella mentale.}

    \emph{Anonimo}
\end{flushright}

\emph{Sabato $7$ agosto, ore $6$:$00$, Sluderno}  
\vspace{2mm}

Suona la sveglia e come di consueto io faccio la mia ginnastica mattutina mentre Tommaso, come al solito, guarda attonito e incredulo della mia energià già a quell'ora.
Facciamo una veloce colazione e prepariamo le ultime cose sulle notre bici.
Saranno tre giorni di nuove emozioni, non abbiamo mai fatto un viaggetto multigiorni con la mtb.

Verso le $7$:$40$ siamo in sella e cominciamo a pedalare tra le monragne della Val Venosta ai piedi dello Stelvio, al quale facciamo l'occhiolino, ci rivederemo tra $2$ giorni.
Giusto qualche km di pianura e inizia subito la salita, prima dolce e successivamente si fa più pendente, dopo una diecina di km iniziano i primi tratti sterrati che ci accompagneranno per la maggior parte dei giorni successivi.
Saliamo verso il \emph{Passo di Slingia} costeggiando una cascata che riempe l'aria di una fitta nebbiolina e ci accompagna sugli ultimi 3 km di salita che sembrano non terminare mai.
Il fondo diventa dissestato e la pendenza diventa importante e dopo un pezzo al $20/25\%$ ci vediamo costretti a smontare dalla bici e spingerla.

Arrivati al passo ci troviamo su un sentiero che sembra uscito direttamente dalle paludi del Signore degli Anelli.
Una fitta nebbia ci impedisce di vedere più in là di una cinquantina di metri e vista l'altitudine di $2309$m la temperatura è scesa fino a $10$ gradi.
Le pioggie della settimana appena passata hanno reso molti sentieri un acquitrino unico e si procede a fatica tra fango e guadi.
Quand'ecco che nel mentre stiamo passando uno dei laghetti Tommaso decide di attarversarlo senza curarsi di bagnarsi ed ad un certo punto pensa di essere finito nei "fanghi mobili".
Per un attimo sprofonda e chiama aiuto.
Dopo un attimo si rende conto che era già sul fondo e continua con l'attraversamento. Si delina così già una costante dei tre giorni: Tommaso con le scarpe bagnate e infangate mentre io cerco di evitare di immergere i piedi nei guadi.
Solo il terzo giorno mi darò per vinto, una volta sola.

Dopo una piccola discesa la nebbia di dirada e, entrati in Svizzera\footnote{Il confine tra Italia e Svizzera è posto poco prima del \emph{Passo di Slingia}.}, possiamo iniziare ad ammirare la magnificenza delle montagne che ci circonda.
Maestose e torreggianti ci incanalano nella valle d'Uina e nelle omonime gole dove è stato scavato dai contrabbandieri un sentiero incastonto su una parete rocciosa alta $800$m.
All'inizio della gola il torrente scorre in prossimità del sentiero, ma dopo qualche decina di metri si getta in varie cascate che rendono il sentiero appeso a metà parete.
Per un paio di km portiamo la bici a mano, il sentiero è stretto, sassaso, ripido e non lascia spazio a errori di manovra.

Al fine del sentiero roccioso si apre la valle d'Uina che porta a \emph{Sur En}, paesello dove passa il fiume \emph{En}\footnote{Il fiume è chiamato anche \emph{Inn} e prosegue per l'Austria, gettandosi infine nel Danubio} il quale nasce tra le pendici del ghiacciaio del Bernina, che raggiungeremo il giorno successivo.
Il sentiero continua in un bosco di abeti e via via che scendiamo diventa più largo e dolce.
I guadi continuano e tra un ponte e l'altro arriviamo in fondo alla discesa nella bassa Valle Engandina.
\vspace{2mm}
\emph{Sabato $7$ agosto, ore $10$:$30$, Sur En} 
\vspace{2mm}

Giunti nel cuore del cantone dei Grigioni, dopo $3h30'$ e circa $30$km pedalati, ci fermiamo per una prima pausa a base di cocacola e paninetti al finicchio con formaggio e salame.

Ristorati ripartiamo per costeggiare dolcemente il fiume \emph{En} e dopo qualche commento sul problema dei movimenti centrali, che mandano in confusione anche il meccanico più eseprto, iniziamo la salita che ci porterà ai $2251$m del \emph{Pass da Costainas}.
La salita, lunga circa $20$km parte abbastanza ripida su asfalto mai poi si addolcisce e diventa di di un gidibile sterrato leggero che si inasprisce sempre di più verso il passo.
Salendo troviamo molti cavalli che e qualche biker che salutiamo mentre chiaccheriamo di frivolezze.

Verso la cima il cielo inizia a diventare plumbeo e si alza un leggero vento.
Arrivati al passo scattiamo qualche foto e dopo una godibile prima parte di discesa ci gettiamo in qualche chilometro di singletrack nel bosco che mettte a dura prova le nostre abilità enduristiche.

\vspace{2mm}
\emph{Sabato $7$ agosto, ore $14$:$30$, Val Mustair} 
\vspace{2mm}

Al fine nella discesa che ci porta nella Val Mustair ci fermiamo un attimo per rifoccillarci con frutta secca e qualche barretta.
Nel frattempo troviamo un cartello che ci rammenta di porre attenzione agli orsi e di non sporcare il bosco.
Facciamo qualche battuta sugli orsi che potrebbero inseguirci mentre affrontiamo le prima rampe della salita successiva che ci porta in $8$km sul \emph{Doss Radond}.
La salita procevede su un fondo leggero ma la pendenza rimane costante al $10\%$ con qualche punta al $15\%$.

Verso il finale un vento freddo soffia dalle montagne sulle nostre facce intralciando la nostra marcia, ma pedalata dopo pedalata arriviamo al $2235$m del passo che ci porta nellla bellissima Val Mora.
Uno splendido sentiero che scende dolcemente verso il confine italiano tra laghi cristallini, prati e qualche ponticello che ci fa costeggiare il torrente che rimbomba nella valle.

Al confine italiano un cippo in mezzo ad una radura ci ricorda che abbiamo lasciato il territorio elvetico.
Ci fermiamo e facciamo una veloce foto tenendo un occhio al cielo, che nel frattempo si è annuvolato ancora di più e non promette nulla di buono.

Arrivati ai laghi di Fraile, poco sopra i laghi di Cancano, inizia a scendere qualche goccia ma la pioggia ci risparmierà fino ai $2285$m del \emph{Passo Alpisella}.
Facciamo appena in tempo ad infilarci la mantellina e inizia una pioggia che ci accompegnerà fino all'agriturismo che abbiamo prenotato.

Dopo qualche chilometro di discesa sono una pioggia leggera e $10$ gradi arriviamo al lago di Livigno dopo percorriamo per qualche km la strada e decidiamo di fermarci per prendere un panino con crudo e formaggio stagionato.
Mentre scegliamo il ripieno dle panino il gestore della bottega ci racconta i vari gradi di stagionatura dei formaggi e dopo aver scelto l'affettato irrompe nella nostra scelta riguardante l'accostamento che abbiamo scelto, chiedo quindi un consiglio ma non vuole rispondermi.
Io ero pronto a consigli ma non me na voluto dare eclissandosi dalla scena.
Tommaso se la ride di gusto, paghiamo e usciamo dal negozzietto.
Nel frattempo ha smesso di piovere così Tommaso si sveste e ci rimettiamo in sella.

Stanchi e affamati ci apprestiamo a fare l'ultima salita verso l'agriturismo posto a $2218$m della Val Federia, che porta il nome della valle e del torrente che scorre verso Livigno.
Tommaso è convinto che sia nulla di più di un falsopiano, ma guardando l'altimetria dal garmin non sembra e procediamo tra strappi al $10/12\%$ e discesette sperando di arrivare il prima possibile all'agriturismo.
Agli ultimi $5$km inizia a piovere di nuovo e man mano che ci avviciniamo alla meta, che sembra non arrivare mai, aumenta di intensità.
L'agriturismo non si vede fino agli ultimi $100$m e questo, complice anche la pioggia, ci disorientà un po'.
Infine, sotto una pioggia intensa, arriviamo all'agriturismo dopo circa $9h$, $111$km e $4100$m.
